\documentclass{article}

\usepackage[left=1.5cm,right=1.5cm,top=1.5cm,bottom=1.5cm,ignoreheadfoot]{geometry}
\usepackage{array}
\usepackage[svgnames,table]{xcolor}

\usepackage[utf8]{inputenc}
\usepackage[english]{babel}

\usepackage{blindtext}
\usepackage{multicol}
\usepackage{color}

\usepackage{comment}
\usepackage{wrapfig}
\usepackage{graphicx}

\usepackage{amsthm,amssymb,amsmath}
\usepackage[shortlabels]{enumitem}

\usepackage{tabularx,ragged2e,booktabs,caption}

\author{Rohit Wason}
\newtheorem{problem}{Problem}
\newtheorem*{solution*}{Solution}
\renewcommand{\theenumi}{\alph{enumi}\)}

\setlength{\columnseprule}{0.5pt} %Separator ruler width
\def\columnseprulecolor{\color{blue}} %Separator ruler colour

\newcommand*{\arraycolor}[1]{\protect\leavevmode\color{#1}}
\newcolumntype{A}{>{\columncolor{teal!50!white}}c}
\newcolumntype{B}{>{\columncolor{LightGoldenrod}}c}

\title{Lesson Plan - Infinite Primes}
\date{4/21/2021}

\begin{document}
\maketitle

\section*{Background story (for class)}
When discussing the Prime numbers, we often assume there are infinitely
many of them. However, this fact is not so obvious. When I first posed this
question to a friend, their first reaction was `aren\textquotesingle t there infinitely many of 
everything?'\\

Many obvious things around us take some careful explaining and the infinitude of Primes
is one of them. Primes have amused humans since early Greeks, and very
little is known about the patterns these numbers exhibit. However, the fact that there
is no limit to Prime numbers has been known to us at least since 300 B.C. While the
earliest proof, due to Euclid, was a little different, 
the present proof is a more commonly used version.\\

Fun fact: Euclid is considered ``Father of Geometry'' and for over 2000 years
his work was considered the \textit{only} form of Geometry!

\section*{Learning Goals}
By the end of this lesson the students will be able to
\begin{itemize}
  \item Understand ``proof by contradiction'',
  \item Generalize the concept of `divisibility',
  \item Demonstrate, using basic vocabulary, that there is no limit to the Prime numbers.
\end{itemize}

\section*{Specific Expectations}
\begin{itemize}
  \item Recognize the divisibility rules for common divisors ($3,4,9,11,\dots$),
  \item Derive a logical conclusion from known facts,
  \item State contrapositive, given a logical statement,
  \item Recognize a contradiction,
\end{itemize}

\section*{Outline of the Course Content}
% \begin{center}
% \sffamily
\arrayrulecolor{white}
\arrayrulewidth=0.5pt
\renewcommand{\arraystretch}{1.5}
\rowcolors[\hline]{3}{.!50!White}{}
\begin{tabular}{A|B}
%   \multicolumn{2}{D}{\bfseries Example table}\\
  \rowcolor{.!70!Black}
  \arraycolor{White}\bfseries Unit Title &
  \arraycolor{White}\bfseries Contents\\
  Unit-1 & Use Truth Tables to discuss logical contradiction. \\
  Unit-2 & Discuss ``contrapositive'' \\
  Unit-3 & Build the proof with class \\
\end{tabular}
% \end{center}

\pagebreak

\section*{Course Content}
\subsection*{Logical Contradiction}
A powerful method of proof that is frequently used in mathematics is 
\textbf{proof by contradiction}. This method is based on he fact that a 
logical statement, $P$ can either be true or false, but never both. The 
idea is to prove that the statement $P$ is true by showing that 
it cannot be false. This is done by assuming that $P$ is false
and proving that it leads to a \textbf{contradiction} (e.g., ``$2$ is odd''
is a logical contradiction).\\

\begin{minipage}{\linewidth}
\centering
\captionof{table}{} \label{tab:title} 
\begin{tabular}{l|l|l}
  \rowcolor{.!70!Black}
  \arraycolor{White}\bfseries $P$ &
  \arraycolor{White}\bfseries $\neg{P}$ &
  \arraycolor{White}\bfseries $P \land \neg{P}$\\
  $T$ & $F$ & $F$ \\
  $F$ & $T$ & $F$ \\
\end{tabular}\par
\bigskip
Class discussion: Does this table show a contradiction?
\end{minipage}

\subsection*{Contrapositive}
The \textbf{contrapositive} of the conditional statement
$P \implies Q$ is the conditional statement
$\neg{Q} \implies \neg{P}$. This is achieved since in the first statement
$P$ was a \textit{necessary condition} for $Q$, the negation of $Q$ in the second 
statement leads us to the negation of $P$.\\

Breakout activity: (come up with examples similar to the following):
\begin{itemize}
  \item If $x=3$, then $x+4=7$.
  \item If $x+4 \ne 7$, then $x\ne 3$.
\end{itemize}

\subsection*{The proof}
The instructor divides the class into small groups
and alternates these steps between the groups, helping 
them derive a conclusion from the prior step(s):
\begin{itemize}
  \item To prove: There are infinitely many Primes.
  \item Suppose the above statement is false.
  \item This implies there are finitely man Prime numbers.
    Let these be $p_1,p_2,p_3,\dots,p_n$.
  \item Now construct a new number, 
    $p=p_1 \times p_2 \times p_3 \times \dots \times p_n + 1.$
  \item Observe: $p$ is larger than any of the primes (discuss examples if needed).
  \item Since $p_1,p_2,p_3,\dots,p_n$ constitute all primes,
    $p$ can't be a prime (define: Composite). Thus $p$ is Composite,
    i.e., is divisible by one of the primes $p_1,p_2,p_3,\dots,p_n$.
  \item But when we divide $p$ by either of $p_1,p_2,p_3,\dots,p_n$,
    we get $1$ as a remainder.
  \item \textbf{This is a contradiction!}
  \item The conclusion: Our assumption in the first step ``there
    are finitely many Prime numbers'', must be wrong.
  \item The required statement is \textit{hence proved.}
\end{itemize}

\end{document}
