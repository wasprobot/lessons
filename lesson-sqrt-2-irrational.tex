\documentclass{article}

\usepackage[left=1.5cm,right=1.5cm,top=1.5cm,bottom=1.5cm,ignoreheadfoot]{geometry}
\usepackage{array}
\usepackage{euler}
\usepackage[svgnames,table]{xcolor}

\usepackage[utf8]{inputenc}
\usepackage[english]{babel}

\usepackage{blindtext}
\usepackage{multicol}
\usepackage{color}

\usepackage{comment}
\usepackage{wrapfig}
\usepackage{graphicx}

\usepackage{amsthm,amssymb,amsmath}
\usepackage[shortlabels]{enumitem}

\usepackage{tabularx,ragged2e,booktabs,caption}

\author{Rohit Wason}
\newtheorem{problem}{Problem}
\newtheorem*{solution*}{Solution}
\renewcommand{\theenumi}{\alph{enumi}\)}

\setlength{\columnseprule}{0.5pt} %Separator ruler width
\def\columnseprulecolor{\color{blue}} %Separator ruler colour

\newcommand*{\arraycolor}[1]{\protect\leavevmode\color{#1}}
\newcolumntype{A}{>{\columncolor{teal!50!white}}c}
\newcolumntype{B}{>{\columncolor{LightGoldenrod}}c}

\title{Lesson Plan - Irrationality of $\sqrt{2}$}
\date{4/25/2021}

\begin{document}
\maketitle

\section*{Background story (for class)}
The set of real numbers, $\mathbb{R}$ is filled with both rational
and irrational numbers. While the rationals are very well understood in our 
vocabulary (think how often we use $1/2$ in our daily usage), the irrationals
are still very mysterious. In fact, it is a proven fact that there are more
(\textit{way many more}) irrationals than there are rationals.\\

But the goal of this lesson is not to prove that fact (it takes some
higher-level math to do that, actually). We will discuss what baffled the 
Pythagoreans in ancient Greece when they discovered that the hypotenuse of a 
``unit-right-triangle'' did not measure up to any number they knew.
(brief discussion: Pythagorean formula for the hypotenuse, $c^2=a^2+b^2$)\\

Fun fact: The Pythagoreans thought that animals have the same rights to live as mankind.

\section*{Supporting concepts}
The following skills are useful to revise while teaching this material:
\begin{itemize}
  \item ``Proof by contradiction'',
  \item Deriving a logical conclusion from known facts,
  \item State contrapositive, given a logical statement,
  \item Recognize the divisibility rules for common divisors ($3,4,9,11,\dots$),
  \item Understand ``greatest common divisor''.
\end{itemize}

\section*{Learning Goals}
By the end of this lesson the students will be able to
\begin{itemize}
  \item Understand that irrational numbers are also real,
  \item Demonstrate, using basic vocabulary, that $\sqrt{2}$ is irrational.
\end{itemize}

\section*{Specific Expectations}
\begin{itemize}
  \item Express a rational number in the $p/q$ form, where $gcd(p,q)=1$,
  \item Recognize a contradiction, while trying to express $\sqrt{2}$ as $p/q$.
\end{itemize}

\section*{Outline of the Course Content}
% \begin{center}
% \sffamily
\arrayrulecolor{white}
\arrayrulewidth=0.5pt
\renewcommand{\arraystretch}{1.5}
\rowcolors[\hline]{3}{.!50!White}{}
\begin{tabular}{A|B}
%   \multicolumn{2}{D}{\bfseries Example table}\\
  \rowcolor{.!70!Black}
  \arraycolor{White}\bfseries Unit Title &
  \arraycolor{White}\bfseries Contents\\
  Unit-1 & Review: Prime Factorization \\
  Unit-2 & Review: $gcd()$ and the fact that for rationals $p/q, gcd(p,q)=1$ \\
  Unit-3 & Build the proof with class \\
\end{tabular}
% \end{center}

\pagebreak

\section*{Course Content}
\subsection*{Review: Prime Factorization}
It's the fundamental fact of arithmetic, that every natural number
($1,2,3,\dots$) can be written as a product of powers of Primes. Examples:
\begin{itemize}
  \item $25=2^0 \cdot 3^0 \cdot 5^2$
  \item $112=2^4 \cdot 3^0 \cdot 5^0 \cdot 7^1$
  \item $1=2^0 \cdot 3^0 \cdot 5^0 \dots$
  \item $13=2^0 \cdot 3^0 \cdot 5^0 \cdot 7^0 \cdot 11^0 \cdot 13^1$ 
    [class discussion: How are Primes unique in this context?]
  \item Class activity: Ask one group of students to think of a number, and 
    another to come up with its Prime factorization.
\end{itemize}

\subsection*{Review: Greatest Common Divisor ($gcd()$)}
Common divisors (factors), as the name suggests, 
are just factors of two (or more) numbers that are common. 
The greatest of them is called the $gcd()$. Examples:
\begin{itemize}
  \item $gcd(14,7)=7$
  \item $gcd(22,33)=11$
  \item $gcd(25,112)=1$ [introduce: co-primes]
  \item $gcd(13,20)=1$ [class discussion: How are Primes unique in this context?]
\end{itemize}

\subsection*{The proof}
The instructor divides the class into small groups
and alternates these steps between the groups, helping 
them derive a conclusion from the prior step(s):
\begin{itemize}
  \item To prove: $\sqrt{2}$ is irrational.
  \item Suppose the above statement is false.
  \item This means that $\sqrt{2}$ can be written as $p/q$ 
    where $gcd(p,q)=1$:
    $$\sqrt{2} = \frac{p}{q}$$
  \item Squaring both sides, we get:
    $$2 = \frac{p^2}{q^2}$$
  \item This implies that:
    $$p^2=2q^2$$
  \item Thus $p^2$ is even. The only way this would be true
    is, if $p$ is even (class discussion: Why is this the case?)
  \item This means that $p^2$ is actually divisible by $4$.
  \item Hence $q^2$ and (using similar argument) $q$ must be even.
  \item But if both $p$ and $q$ are even, $gcd(p,q)=2\ne{1}$.
  \item \textbf{This is a contradiction!}
  \item The conclusion: Our assumption in the first step that 
    $\sqrt{2}$ can be written as $p/q, gcd(p,q)=1$, or that
    $\sqrt{2}$ is rational, must be wrong.
  \item The required statement is \textit{hence proved.}
\end{itemize}

\end{document}
